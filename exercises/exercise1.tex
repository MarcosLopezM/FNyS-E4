\documentclass[./../main.tex]{subfiles}
\graphicspath{{img/}}

\begin{document}
    \begin{exercise}
        Un reactor de \qty{100}{\MW} consume la mitad de su combustible en 3 años. ¿Cuánto \ch{^{235}U^{92}} contiene el reactor?

        \begin{solution}
            Sabemos que la potencia está dada por

            \begin{align*}
                P &= \dfrac{E}{t},\\
                \implies E &= P \cdot t.
            \end{align*}

            Así,

            \begin{align*}
                E &= \qty{100}{\MW} \cdot \qty{9.46e7}{\s},\\
                \Aboxedsec{E &= \qty{9.46e15}{\J}.}
            \end{align*}

            Recordando que

            \begin{equation*}
                E = mc^{2},
            \end{equation*}

            pero como la energía obtenida es para cuando el reactor ha consumido la mitad de su combustible, así que

            \begin{align*}
                E &= \dfrac{m}{2}c^{2},\\
                \implies m &= \dfrac{2E}{c^{2}}.
            \end{align*}

            Por lo que la cantidad de combustible es de

            \begin{align*}
                m &= \dfrac{2(\qty{9.46e15}{\J})}{(\qty{3e8}{\m\per\s\squared})^{2}},\\
                \Aboxedmain{m &\approx \qty{0.21}{\kg}.}
            \end{align*}
        \end{solution}
    \end{exercise}
\end{document}