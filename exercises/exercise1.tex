\documentclass[./../main.tex]{subfiles}
\graphicspath{{img/}}

\begin{document}
    \begin{exercise}[Estructura hiperfina]
        Primero, explica el origen físico de la estructura hiperfina. Explica por qué se espera que el efecto de la estructura hiperfina sea del orden de \(\tfrac{m}{M_{p}}\) menor que el de la estructura fina. Aquí \(m\) es la masa del electrón y \(M_{p}\) la del protón.
        
        Ahora, recuerda que en clase se demostró que la corrección de primer orden de la energía debida a la estructura hiperfina está dada por

        \begin{equation*}
            \change{E} = \tfrac{C}{2}\left[F(F + 1) - I(I + 1) - j(j + 1)\right]
        \end{equation*}

        en donde

        \begin{equation*}
            C = \dfrac{\mu_{0}}{4\pi}4g_{I}\mu_{B}\mu_{N}\dfrac{1}{j(j + 1)(2\ell + 1)}\left(\dfrac{Z}{a_{\mu}n}\right)^{3}.
        \end{equation*}

        \begin{enumerate}
            \item (Valor: 1pt) - Utilizando este resultado demuestra que en un \textbf{mismo} multiplete hiperfino, la separación entre dos niveles hiperfinos \textbf{subsecuentes} está dada por
            
            \begin{equation*}
                \change{E}(F) - \change{E}(F - 1) = CF.
            \end{equation*}

            Notemos que esta separación, conocida como \emph{separación hiperfina}, es proporcional a \(F\). Esta propiedad es un ejemplo de los que se conoce como \emph{regla del intervalo}.

            \item (Valor: 1pt) - En un \textbf{mismo} multiplete hiperfino, las componentes hiperfinas más separadas tienen como valores del número cuántico a \(F\) a \(F_{\text{máx}} = I + j\) y \(F_{\text{mín}} = \abs{I - j}\). Demuestre que la separación entre estos dos niveles hiperfinos está dada por
            
            \begin{equation*}
                \fdif{E} \equiv \adif{E}(F_{\text{máx}}) - \adif{E}(F_{\text{mín}}) = 2C \mul \begin{dcases*}
                    j(I + \tfrac{1}{2}) & si \(j \leq I\)\\
                    I(j + \tfrac{1}{2}) & si \(j \geq I\)
                \end{dcases*} 
            \end{equation*}
            
            \item (Valor: 1pt) - Considera ahora el caso del átomo de hidrógeno, es decir, el caso en el que el núcleo que interactúa con el electrón está compuesto apenas por un protón. El número cuántico de spin del protón es \(I = \tfrac{1}{2}\). En este caso, el estado fundamental del sistema, el nivel \(1s_{1/2}\) se divide en dos estados hiperfinos con \(F = 0\) y \(F = 1\). Demuestra que la separación entre estos dos estados hiperfinos es
            
            \begin{equation*}
                \fdif{E} = \dfrac{\mu_{0}}{4\pi}\dfrac{16g_{I}\mu_{B}\mu_{N}}{3a_{0}^{3}}.
            \end{equation*}

        \item (Valor: 0.5pt) - Calcula la diferencia de energía (en unidades de \unit{\eV}) del desdoblamiento hiperfino del estado fundamental del átomo de hidrógeno (para el protón \(g_{I} \simeq \num{5.588}\)), ¿cuál es la frecuencia \(\nu\) (en \unit{\MHz}) y la longitud de onda \(\lambda\) (en \unit{\cm}) de la transición entre estos niveles?
        
        \item (Valor: 0.5pt) - El núcleo del átomo de deuterio está compuesto por un protón y un neutrón. Para este núcleo \(I = 1\) y \(g_{I} \simeq \num{0.857}\). Calcula el cociente entre los desdoblamientos hiperfinos del estado fundamental de los átomos de hidrógeno y deuterio.
        \end{enumerate}
    \end{exercise}
\end{document}