\documentclass[./../main.tex]{subfiles}
\graphicspath{{img/}}

\begin{document}
    \begin{exercisex}[Velocidades en el Modelo de Electrones Libres]
        Considera un metal en donde los electrones de conducción son descritos por el modelo de electrones libres. Recordemos que en este modelo, la velocidad promedio de los electrones en la superficie de Fermi está dada por la velocidad de Fermi \(v_{F} = \slashfrac{\hbar k_{F}}{m}\), en donde \(k_{F}\) es el vector de onda de Fermi.

        \begin{enumerate}
            \item (Valor: +1pt) - Usando los resultados obtenidos en clase, demuestra que la \textbf{velocidad de arrastre} de un electrón en presencia de un campo eléctrico \(\vect{E}\) está dada por
            
            \begin{equation*}
                \vect{v}_{a} = -\dfrac{\sigma\vect{E}}{n\e}
            \end{equation*}

            en donde \(\sigma\) es la conductividad eléctrica. Demuestra también que \(\sigma\) en términos del camino libre medio \(\ell\) está dada por

            \begin{equation*}
                \sigma = \dfrac{n\e^{2}\ell}{mv_{F}}.
            \end{equation*}

            \item (Valor: +1pt) - Considera un alambre hecho de cobre:
                \begin{enumerate}[label = (b.\arabic*)]
                    \item Calcula el valor de \(v_{a}\) y de \(v_{F}\) suponiendo que la temperatur del alambre es de \qty{300}{\kelvin} y se le aplica un campo eléctrico cuya magnitud es \(E = \qty{1}{\V\per\m}\). Comenta sobre cómo se comparan ambas velocidades.
                    \item ¿Qué magnitud tendría que tener dicho campo eléctrico para que \(v_{a}\) y \(v_{F}\) fueran iguales? Escribe tu resultado en unidades de \unit{\V\per\m}.
                    \item Estima el valor del camino libre medio \(\ell\) para el cobre a \qty{300}{\kelvin} y compara este valor con el espaciamiento promedio de los átomos del metal.
                \end{enumerate}

                Información útil para resolver este problema: el cobre es un metal monovalente, lo que significa que hay un electrón libre por cada átomo ( o sea, cada átomo del metal dona un electrón de conducción). La densidad de átomo del cobre es \(n = \qty{8.45e28}{\atoms\per\m\cubed}\). La conductividad eléctrica del cobre a \qty{300}{\K} es \(\sigma = \qty[per-mode=symbol]{5.9e7}{\per\ohm\per\m}\)
        \end{enumerate}
    \end{exercisex}
\end{document}