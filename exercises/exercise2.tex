\documentclass[./../main.tex]{subfiles}
\graphicspath{{img/}}

\begin{document}
    \color{blue}
    \begin{exercise}[Extra (valor: +2pt): Configuración electrónica de átomos multielectrónicos]
        Utilizando el Principio de Aufbau y la Regla de Madelung, construye la configuración electrónica de los siguientes átomos:

        \begin{enumerate}[threecol]
            \item Cobre (\ch{Cu}, \(Z = 29\)).
            \item Arsénico (\ch{As}, \(Z = 33\)).
            \item Neodimio (\ch{Nd}, \(Z = 60\)).
            \item Oro (\ch{Au}, \(Z = 79\)).
            \item Radón (\ch{Rn}, \(Z = 86\)).
            \item Neptunio (\ch{Np}, \(Z = 93\)).
        \end{enumerate}

        Ahora compara tus respuestas con las configuraciones electrónicas reportadas en el libro ``Physics of Atoms and Molecules''  de B.H. Bransden y C.J. Joachain (en la primera edición de este libro véase la Tabla 7.2 de la página 302).
        ¿Coinciden tus respuestas con las reportadas en el libro? En los casos en los que no coincidan explica cualitativamente porqué.
    \end{exercise}
\end{document}