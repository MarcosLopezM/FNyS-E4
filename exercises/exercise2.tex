\documentclass[./../main.tex]{subfiles}
\graphicspath{{img/}}

\begin{document}
    \begin{exercise}
        Como el tiempo de vida media de \ch{^{235}U} (\qty{7.13e8}{\years}) es menor al tiempo de vida media de \ch{^{238}U} (\qty{4.51e9}{\years}), la abundancia de \ch{^{235}U} ha ido decreciendo en la Tierra. ¿Hace cuánto tiempo la abundancia isotrópica del \ch{^{235}U} era igual a \SI{3}{\percent}? Este porcentaje es el enriquecimiento que se usa en algunas plantas nucleares.

        \begin{solution}
            La actividad de un material al tiempo \(t\) está definida por

			\begin{equation}
				\mathcal{A}(t) = \mathcal{A}_{0}\e^{-\lambda t}.
				\label{eq:activity}
			\end{equation}

            Sabemos que el \ch{^{238}U} es más abundante que el \ch{^{235}U}, tal que una muestra uranio contiene un \qty{99.28}{\percent} de \ch{^{238}U} y un \qty{0.72}{\percent} de \ch{^{235}U}. Y puesto que queremos determinar cuando la abundancia era del \qty{3}{\percent} para el \ch{^{235}U} y, por ende, la abundancia del \ch{^{238}U} era del \qty{97}{\percent}, su expresión para la actividad es

            \begin{equation*}
                \mathcal{A}(t) = \dfrac{\num{0.72}}{\num{99.28}} = \dfrac{3}{97}\e^{-(\lambda_{1} - \lambda_{2})t},
            \end{equation*}

            donde \(\lambda_{1}\) es la constante de desintegración del \ch{^{235}U} y \(\lambda_{2}\) es la constante de desintegración del \ch{^{238}U}. 

            Resolviendo para \(t\) se obtiene que

            \begin{align*}
                t &= \dfrac{\ln\left(\tfrac{\num{99.28}}{\num{0.72}}\right)}{\ln\left(\tfrac{97}{3}\right)}\dfrac{1}{\lambda_{1} - \lambda_{2}},\\
                &= \dfrac{\ln\left(\tfrac{\num{99.28}}{\num{0.72}}\right)}{\ln\left(\tfrac{97}{3}\right)}\dfrac{1}{\qty{9.7e-10}{\years} - \qty{1.5e-10}{\years}},\\
                \Aboxedsec{t &= \qty{1.73e9}{\years}.}
            \end{align*}

            Por lo que aproximadamente hace \num{1.73} billones de años la abundancia isotrópica del \ch{^{235}U} era del \SI{3}{\percent}.

        \end{solution}
    \end{exercise}
\end{document}