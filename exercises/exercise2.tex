\documentclass[./../main.tex]{subfiles}
\graphicspath{{img/}}

\begin{document}
    \begin{exercise}
        Si la masa del Sol es de \qty{e29}{\kg}, y su vida total es de \num{e9} años, ¿qué potencia disipa al año?

        \begin{solution}
            Sabemos que la energía que se libera por en el ciclo \ch{p-p} es de \qty{24.68}{\MeV}, que en \unit{J} es

            \begin{empheq}[box = \secbox]{equation*}
                E_{\text{pp}} = \qty{3.95e-12}{\J}.
            \end{empheq}

            Y la cantidad de núcleos se obtiene a partir de

            \begin{align*}
                N &= (\num{e29}\cdot\qty{1000}{\g})\cdot \dfrac{\qty{6.023e23}{\per\mol}}{\qty{1}{\g\per\mol}},\\
                \Aboxedsec{N &= \num{6.023e56}.}
            \end{align*}

            Entonces, la energía liberada es de

            \begin{align*}
                E &= (\qty{3.95e-12}{\J})(\num{6.023e56}),\\
                \Aboxedsec{E &= \qty{2.38e45}{\J}.}
            \end{align*}

            Finalmente, para obtener la potencia disipada por año,

            \begin{align*}
                P &= \dfrac{\qty{2.38e45}{\J}}{\qty{3.15e16}{\s}},\\
                \Aboxedmain{P &= \qty{7.5417e7}{\W}.}
            \end{align*}
        \end{solution}
    \end{exercise}
\end{document}