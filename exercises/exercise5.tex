\documentclass[./../main.tex]{subfiles}
\graphicspath{{img/}}
\begin{document}
	\begin{exercise}
		Si cada fisión del \ch{^{235}U} genera en promedio 2.5 neutrones de energías térmicas (aproxima a \qty{1}{\eV}) ¿qué cantidad de ese combustible es necesario para recibir una dosis alta de (\qty{5}{\sievert}) en una persona de \qty{80}{\kg} de peso (solo proveniente de neutrones)?

		\begin{solution}
			Sabemos que la dosis absorbida está dad por la energía absorbida por unidad de masa, tal que
			
			\begin{equation*}
				D \simeq \dfrac{E}{m}.
			\end{equation*}

			Por lo que la energía absorbida es

			\begin{align*}
				E &\simeq D \cdot m,\\
				&\simeq (\qty{5}{\sievert})(\qty{80}{\kg}),\\
				&\simeq (\qty[per-mode=symbol]{5}{\J\per\kg})(\qty{80}{\kg}),\\
				\Aboxedsec{E &\simeq \qty{400}{\J}.}
			\end{align*}

			Para determinar la cantidad del combustible, o el número de neutrones que se necesitan, debemos determinar la energía que se libera por cada fisión. Para ello, sabemos que la energía liberada por cada fisión es de \qty{1}{\eV} por neutrón, por lo que la energía liberada por cada neutrón es de

			\begin{align*}
				E_{n} &\simeq \dfrac{\qty{1}{\eV}}{2.5},\\
				E_{n} &\simeq \qty[per-mode=symbol]{6.4e-20}{\J\per\neutron}.
			\end{align*}

			Entonces, la cantidad de neutrones que se necesitan es

			\begin{align*}
				N_{n} &\simeq \dfrac{E}{E_{n}},\\
				&\simeq \dfrac{\qty{400}{\J}}{\qty{6.4e-20}{\J\per\neutron}},\\
				\Aboxedmain{N_{n} &\simeq \qty{6.25e21}{\neutrons}.}
			\end{align*}
		\end{solution}
	\end{exercise}
\end{document}
