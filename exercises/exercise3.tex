\documentclass[./../main.tex]{subfiles}
\graphicspath{{img/}}
\begin{document}
	\begin{exercise}
		¿Qué masa de hidrógeno necesitas para generar \qty[inter-unit-product=]{1}{\MWD}?

		\begin{solution}
			Del ciclo \(p-p\) sabemos que al colisionar dos hidrógenos la energía que se libera es de \(\qty{0.42}{\MeV} \simeq \qty{7.79e-25}{\MWD}\). Partiendo de la expresión para la energía liberada

			\begin{equation}
				E \simeq (\qty{7.79e-25}{\MWD})N,
				\label{eq:energy_dissipated}
			\end{equation}

			con \(N\) la cantidad de núcleos, \idest

			\begin{equation*}
				N = m \cdot \dfrac{N_{A}}{A}.
			\end{equation*}

			Sabemos que la energía generada es de \qty{1}{\MWD}, entonces de \cref{eq:energy_dissipated}

			\begin{align*}
				\qty{1}{\MWD} &\simeq (\qty{7.79e-25}{\MWD})(m \cdot \qty{6.023e23}{\per\g}),\\
				m &\simeq \dfrac{\qty{1}{\MWD}\cdot\qty{1}{\g}}{(\qty{7.79e-25}{\MWD})(\num{6.023e23})},\\
				\Aboxedmain{m &\simeq \qty{2.13}{\g}.}
			\end{align*}
		\end{solution}
	\end{exercise}
\end{document}
