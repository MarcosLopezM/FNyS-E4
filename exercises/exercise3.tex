\documentclass[./../main.tex]{subfiles}
\graphicspath{{img/}}

\begin{document}
    \begin{exercise}[Efecto Stark]
        Cuando un átomo se encuentra en presencia de un campo eléctrico constante \(\vect{E}_{s}\) los niveles de energía se desplazan, un fenómeno conocido como ``\textbf{Efecto Stark}''. En este problema analizamos únicamente el efecto Stark en los estados \(n = 1\) y \(n = 2\) del átomo de hidrógeno. Supongamos que el campo eléctrico apunta en la dirección \(z\), de manera que la energía potencial del electrón es

        \begin{equation*}
            \op{W}{S} = -\e E_{s}z = \e E_{s}r\cos\theta.
        \end{equation*}

        Considera que \(\op{W}{S}\) es una perturbación sobre el átomo de hidrógeno de Bohr (es decir, no es necesario que consideres la estructura fina e hiperfina). También puedes ignorar al spin del electrón en este problema.

        \begin{enumerate}
            \item (Valor: 1pt) - Demuestra que, a primer orden, esta perturbación no altera al estado fundamental.
            \item (Valor: 3pt) - El primer estado excitado presenta degeneración cuádruple: \(\psi_{200}\), \(\psi_{211}\), \(\psi_{210}\) y \(\psi_{21-1}\). Usando teoría de perturbaciones para el caso degenerado, determina la corrección de la energía a primer orden. ¿En cuántos niveles se desdobla la energía \(E_{2}\)?
            
            \color{blue}
            \item Inciso extra (+2.5pt) - ¿Cuál es la mejor base para realizar el cálculo del inciso (b)? Encuentra el valor esperado del operador de momento dipolar eléctrico (\(\vect{P} = -\e\vect{r}\)) para cada uno de los miembros de dicha base que son pertinentes para este problema.
            
            \textbf{Pista}: La ``mejor base'' es justamente aquella que diagonaliza a la matriz \(\op{W}{S}\).
        \end{enumerate}
    \end{exercise}
\end{document}